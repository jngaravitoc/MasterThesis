% Introduction

\chapter{Introduction} % Main chapter title

\label{sec:intro} % For referencing the chapter elsewhere, use \ref{Chapter1} 

\lhead{\emph{Introduction}} % This is for the header on each page - perhaps a shortened title

\section{\emph{Motivation}}

It is astonishing how much we can learn from observing and 
conteplating the Universe. With the discovery of the planets...

Astronomy is unique in the sense that there is only one Universe 
to observe. Due to the expanding nature of the Universe and 
that the speed of light is finite, observations of the farther 
regions of the Universe are also from the youngest stages of the Universe.
This allows to observe the evolution of the Universe during redshift.

There are two techniques to observe the Universe, photometry and spectroscopy.  
Photometric observations measures the flux astronomical objects in different filters. This allows to dereive quantites such as the luminosity and the temperature. While spectroscopic observations would let study in deep detail
the spectra of anstronomical objects, in particular the spectra of 
galaxies reveals properties such as: the popuation of stars in the galaxy, the
redshift, the elements that constitute this objects, the characteristics 
of the instersetellar medium and many more.

In particular the Lyman$\alpha$ emission line is a usefull 
tool to detect star forming galaxies at high redshift, specially 
at $z>2$ which is when the line is shifted to the optical frame. But 
as discussed in X the morphology of the line is substantially affected
by the gas kinematics \& distribution, the interstellar gas (IGM)
 and by the dust. 

Understanding the morphology of the \ly line is a crusial endeavour 
to understand deeply the properties of galaxies at high redshift. To this aim 
computer simulations are extremely usefull, models of this properties 
can be simulated to creat synthetic profiles that can be compared
with the observations. 

This is an important era to study this high redshift galaxies. Computer 
simulations techniques have been imporved and more powerfull facilites are
now aveilable. Also a new generation of telescopes are being constructed,
this telescopes would have the power to observe more deeply in the sky, 
and new high redshift galaxies are going to be discovered. This would 
improved our understanding of the high redshift Universe. 

\section{\emph{Historical Remarks}}

The emission of the Lyman $\alpha$ line in galaxies was first predicted by \citep{PartridgePeebles} in their work they suggest that 




\section{\emph{The Lyman $\alpha$ line in astronomy}}

The Lyman $\alpha$ (Here after \ly) emission line is a consecuence of the transition 
from the first excited level to the ground level of the electrons
in the Hydrogen atom. When the electron undergoes in such transition 
a photon is emitted with an energy of $13.6eV$ which corresponds to 
a rest wavelenght of $1215.67 \AA$. 

The main sources of the \lya emission is star formoing galaxies, 

\subsection{Resonant scattering}

Hydrogen is the most abundant element in the Universe furthermore
the Lyman$\alpha$ line is a very common line. However the transmission of this
Lyman$\alpha$ photons through the Hydrogen clouds it's not trivial. There 
are two main aspects that heavily influce the morphology of the line.\\

Imagine that for some reason there is one Hydrogen atom which emits 
a \ly photon in the middle of a cloud of Hydrogen atoms which are in the
ground state. The \ly photon will be abosrved and re-emitted from 
the Hydrogen atoms. This process will end up in a random walk in the space,
and the photon can escape the cloud at any point in the surface of it.\\

In the previous situation all the Hydrogen atoms were in rest, if the atoms
present proper motions there would be a doppler shift. If an atom have a velocity 
$v$ an emit a \ly photon in the direction opposite to the movement the \ly photon 
would be redshifted by an ammount r  

 

\subsection{Units, Quantities \& Definitions}

Through this work we will use some units and quantities that perhaps the reader 
may not be familiar with. Here we define this quantities. 

\subsection{Optical depth and Column Density}

\subsection{\ly line profile units} 

The frequency $\nu_{\alpha}$ of a \ly photon in rest frame is 
$2.46\times 10^{15}Hz$, 
there are two main units in which the frequency of the \ly alpha photons 
may be presented in the literature.  

The first is a dimensionless vairable $x$ defined as:
\begin{equation}\label{eq:x}
x   \equiv \dfrac{(\nu -\nu_{\alpha})}{\Delta \nu_D}
\end{equation}

Where $\nu$ is the frecuency in the observer frame,  
$\nu_D$ is the broadening of the line due to the termal 
velocity of the Hydrogen atoms $v_{th}$. Which can be derived 
assumming that the Hydrogen gas is in thermal equilibrium 
and follow a Maxwellian distribution.

\begin{equation}
\dfrac{m_H v_{th}}{2} = K_B T 
\end{equation} 

Where $m_H$ is the Hydrogen atom mass, $K_B$ the Boltzmann constant and $T$ 
the temparature, the expresion for $v_{th}$ is then:

\begin{equation}
v_{th} = \sqrt{\dfrac{2 K_B T}{m_H}} = 128.5 T^{1/2}\dfrac{m}{s}
\end{equation}

The Doppler shift due to $v_{th}$ would be:

\begin{equation}
\nu'= (1 - \dfrac{\vec{v_{th}}\cdot\vec{n}}{c})\nu_{\alpha}
\end{equation}

Which can be expressed in terms of $\Delta \nu_D$ as: 

\begin{equation}
\nu_{\alpha} - \nu' = \Delta\nu_D =  \dfrac{\vec{v_{th}}\cdot\vec{n}}{c}
\end{equation}

It is also common to express the line profile in terms of the velocity $V$
making use of Eq.\ref{eq:x}:

\begin{equation}
V = xv_{th} = \dfrac{\nu - \nu_{alpha}}c
\end{equation}

\section{\emph{Is the \ly emission line the most prominent line in the Universe?}}

There were 25 years since the prediction of the \ly line to the first observation by \verb+\citep{Djorgovski and Thompson}+, 

1. Not so popular, why?, dust, gas kinematics

2. what about high z?

3. Symmetry of the line---morphology

\section{\emph{Analytical Models}}

In the path of understanding the \ly line profile, ideal 
situations have been studied in order to obatin analytical 
models of the profiles. This models provide the theory that
would let develope the modern codes enable to 
This models are the foundations. In this section the most relevant 
analytical models are explained in the chronological in whcih 
they have been developed. 
 

\section{\emph{Simulated models \& techniques}}

In the Universe most of the galaxies have, this charactersitics
can not be implemented. There are two main paths in which the 
simulations have done. The first approach is by implementing 
the properties of galaxies in the codes, such as: The geometry
of the gas, the kinematics of the gas and the dust. Meanwhile 
some other model a realistic galaxy in the sense of gas geometry
and kinematics using hydrodynamics techniques. In this section 
we briefly descript this two techniques. 

\subsection{Monte-Carlo codes}

Monte-Caro techniques are used ...

\subsection{Hydrodynamic codes} 

\subsection{Models}

Using Monte-Carlo codes multiple geometries and kinematics has been 
studied. The outflows of gas in the galaxy due to supernova explotions
has been studied by X, the main result is outflows lead to a asymmetric
profiles ...

\section{\emph{This Thesis}}

In this thesis we study the effect of rotation, an intrinsic
 characteristic of galaxies, on the morphology
of the \lya outcoming profile, to this aim we implement a solid body
rotation model in the radiative transfer code \verb+CLARA+ \citep{CLARA}.
All the codes used for the analysis of this work is reproducible
and public available in \href{https://github.com/jngaravitoc/RotationLyAlpha}{github}.

\subsection{Summary of the thesis}

A lot of progress has been done in modelling the \ly line using radiative
trasnfer codes. Properties of the gas kinematics such as outflows/inflows. 
Geometries such as slabs, spheres, cavities and the propagation of \ly photons
in a clumpy media. Dust effects XX. In this thesis we study the effect 
of rotation in the morphology of the \ly profile.  

We model a galaxy as an sphere, with an homogeneous mixture of gas and dust. 
We took the rotation velocity, the optical depth, the viewing angle as free
parameters of the model. We also have two different sources of \ly photons:
A central distribution in the galaxy and an homogeneous distribution. 

We quantify the effect of rotation with the following characteristics: 
Escape fraction of \ly photons, the width of the \ly line, the average number
of scatterings of the \ly photons and with the position of the \ly line maxima.

Our main finding is that rotation do have an important effect on the morphology 
of the \ly line. Specially in the width of the line and in the position of the 
maxima. While the average number of scatterings  and the scape
fraction remain constant.

The line broadens proportional to the rotation velocity, and also the flux in 
the middle of the line increases with rotation.

The axys of rotation breaks the symmetry of the system, Althogh 
observers in different viewing angle with respect to the rotation axys
observe the same amount of flux of the \ly profile. This result
lead us to find an approximate analytic solution to model rotation.
 
 
With these results an approximated analytical solution was derived, 
taking into the account that the radiative transfer inside de gas
cloud is exactly as in the static case. In the sphere surface
a Doppler shift due to the difference velocity of an external 
obsever and the surface have to be taken into account.  


\subsection{The importance of modelling the effect of gas bulk rotation}

Modelling the effect of rotation would play a progress in the effort
of modelling realistic galaxies. Quantifying the effect of roation 
on the \ly line would lead to distinguish the kinematic state
of the galaxy. Future projects  

Also an analytic solution 

\subsection{Future work}

Fitting \ly line to meassure the rotational velocities of galaxies.

The effect of outflows and rotation...
