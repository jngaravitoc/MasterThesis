% Introduction

\chapter{Introduction} % Main chapter title

\label{sec:intro} % For referencing the chapter elsewhere, use \ref{Chapter1} 

\lhead{\emph{Introduction}} % This is for the header on each page - perhaps a shortened title

\section{\emph{Motivation}}

It is astonishing how much we can learn from observing and 
contemplating the Universe. All this knowledge of the Universe
led to a better comprehension of the dynamics of the Universe itself 
and how these components such as galaxies, interstellar/intergalactic
medium and stars have been evolved in time. These theories don't have
just an academic impact society has faced different changes of 
paradigm due to these scientific achievements. And hopefully this
knowledge of nature can help us in the path to a modern 
sustainable society with consciousness of our place and role 
in the Universe.  

Astronomy is unique in the sense that there is only one Universe 
to observe. Due to the expanding nature of the Universe and 
that the speed of light is finite, observations of the farther 
regions of the Universe are also from the youngest stages of the Universe.
This allows to observe the evolution of the Universe during redshift.

There are two techniques to measure the radiation coming from the 
Univserse; photometry, spectroscopy and polarimetry. Photometric observations measure 
the flux of astronomical objects in different filters. This allows to derive quantities 
such as the luminosity and the temperature. On the other hand spectroscopic observations 
allow to study in deep detail
the spectra of astronomical objects, in particular the spectra of 
galaxies reveal properties such as the population of stars in the galaxy, the
redshift, the elements that constitute these objects, the characteristics 
of the insterstellar medium and many more.

In particular the Lyman$\alpha$ emission line is a useful 
tool to detect star forming galaxies at high redshift, specially 
at $z>2$ which is when the line is shifted to the optical frame. But 
as discussed in \S\ref{sec:resonant} the morphology of the line 
is substantially affected
by the gas kinematics and distribution, the interstellar gas (IGM)
 and by the dust. 

Understanding the morphology of the \ly line is a crucial endeavour 
to understand deeply the properties of galaxies at high redshift. To this aim 
computer simulations are extremely useful, models of these properties 
can be simulated to create synthetic profiles that can be compared
with the observations. 

This is an important era to study these high redshift galaxies. Computer 
simulations techniques have been improved and more powerful facilites are
now available. Also a new generation of telescopes are being constructed,
these telescopes will have the power to observe more deeply into the sky, 
and new high redshift galaxies are going to be discovered. All these  
improve our understanding of the highredshift Universe. 



\section{\emph{Historical Remarks}}

The emission of the \ly line in galaxies was first predicted by 
\citep{PartridgePeebles} in their work they suggest that the \ly luminosity
could increase the luminosity of the galaxies at high redshift, and that 
it could be detected by the telescopes. 

By the same epoch the radiative transfer theory in \ly systems started to be
 studied \citep{Osterbrock62, Adams72, Harrington73, Neufeld90}
 it can be seen as a diffusion process if it is carried out in 
 an optically thick medium. It means that \ly photons propagating 
in a medium are diffusing in space and in wavelenght from the line center.   
Because of the complexity of these systems an analytical solution 
can only be carried out in simplified systems. 


Almost tree decades later \citep{DjorgovskiThomson92} detected the first \ly 
emitter galaxy (LAE) since then hundreds of galaxies has been  
observed at redshifts $z>2$. There is a confirmed LAE at $z=8.6$ 
by \citep{Lenhert2010} and candidates up to $z\sim12$ by \citep{Brammer13}. 
The \ly line has the potential to be a tool to confirm galaxies at the highest 
redshifts. 

Despite all these observational efforts to fully understand the radiative
transfer (RT) of the \ly line numerical simulations started to be performed.
\citep{DijkstraKramer, Laursen09, Verhamme06, CLARA}
in order to understand more realistic situations.

More recently various teams \citep{Mas-Hesse09,LARS} have
observed nearby galaxies with the Hubble Space Telescope with the
aim to have better spectra and images of the HI distribution 
and kinematics. These new data now can be compared with the numerical 
models developed.

 

\section{\emph{The Lyman $\alpha$ line in astronomy}}\label{sec:lyuses}

The Lyman $\alpha$ (hereafter Ly$\alpha$) emission line is a consequence of the 
transition 
from the first excited level to the ground level of the electron
in the Hydrogen atom. When the electron undergoes such transition 
a photon is emitted with an energy of $13.6eV$ which corresponds to 
a rest wavelenght of $1215.67 \AA$. 

This transition is very common in the Universe due to the high abundance
of hydrogen. Two main mechanisms end up in \ly radiations: Recombination 
processes and collisional events of Hydrogen atoms, UV stellar radiation
from star regions (Recombination), Gravitational 
cooling (Collisional) and UV background radiation (Recombination). The interested
reader could read \citep{LaursenPhD} and references within for
more details. 

The detection of the \ly line has been broadly studied and has multiple 
applications in extragalactic astronomy:

\begin{itemize}
\item {\bf{Detection of high $z$ galaxies:}} One of the most used 
methods to detect high redshift galaxies is finding the \ly emission 
line using narrow band imaging or spectroscopy. With these methods 
hundreds of galaxies has been detected. This 
allow to study the properties of the high redshift Universe such as
the large scale structure. 
 
\item {\bf{Galaxy formation and evolution:}} Observed LAEs at different
reshifts has allowed to derive the Lumimosity Functions (LFs) of LAEs, 
for this aim a good knowledge of the escape fracition of 
photons is requiered $f_{esc}$ see \S \ref{sec:ef} for details on the escape 
fraction of \ly photons. As consequence the observed LAEs LFs enhance a
better understanding of the galaxy formation processes. 

\item {\bf{Reionization:}} The Epoch of Reionization (EoR) is an important
period of the Universe, in which the neutral cold HI gas reionize to 
become a hot gas. There is evidence from CMB measurements that 
the EoR must start at $z >> 11$ and should end up at $z\sim5$. During this 
epoch the intergalactic medium (IGM) became opaque to the \ly radiation 
outcoming from the galaxies. LAEs has also shown to be 
very useful in constraining the time at which the EoR ends. 
For a complete recent review please see \citep{review} and references
within.
 
\end{itemize}


\section{Units, Quantities \& Definitions}

Through this work we will use some units and quantities very popular in 
this area. For clarity here we briefly define and review this 
quantities and units.

\subsection{Optical depth and Column Density}

The optical depth $\tau$ is a measure of the "transparency" of the medium
in which the photons are propagating, $\tau$ is defined as:

\begin{equation}
d\tau = \alpha dr
\end{equation}

Where $\alpha$ is the absorption coefficient defined as:

\begin{equation}
\dfrac{dI}{dr} = -\alpha I
\end{equation}

From these definitions we can now define the "thickness" of the medium.
If $\tau > 1$ we said that the medium is optically thick and this means
that a photon with frequency $\nu$ cannot go through the medium without been
absorbed. If $\tau < 1$ the medium is optically thin and the photon
could be pass through the medium without been absorbed.

The optical depth measure the distance from the source
of photons to the surface of the medium. It is also very common 
to express the amount of gas between the \ly source and the surface
in terms of the Hydrogen column density 
$N_{HI}$ which is related to the optical depth by $N_{HI} = \tau/\sigma$
where $\sigma$ is the Hydrogen atom cross section. 

\subsection{\ly line profile units}

The frequency $\nu_{\alpha}$ of a \ly photon in rest frame is
$2.46\times 10^{15}Hz$,
there are two new variables $x, V$ in which the frequency of 
the \ly alpha photons may be presented in the literature.

The first is a dimensionless variable $x$ defined as:

\begin{equation}\label{eq:x}
x   \equiv \dfrac{(\nu -\nu_{\alpha})}{\Delta \nu_D}
\end{equation}

Where $\nu$ is the frequency in the observer frame,
$\nu_D$ is the broadening of the line due to the thermal
velocity of the Hydrogen atoms $v_{th}$. This velocity can be derived
assuming that the Hydrogen gas is in thermal equilibrium
and follows a Maxwellian distribution. In thermal equilibrium
we get:

\begin{equation}
\dfrac{m_H v_{th}}{2} = K_B T 
\end{equation}

Where $m_H$ is the Hydrogen atom mass, $K_B$ the Boltzmann constant and $T$
the temparature in Kelvins, the expression for $v_{th}$ is then:

\begin{equation}
v_{th} = \sqrt{\dfrac{2 K_B T}{m_H}} = 128.5 T^{1/2}\dfrac{m}{s}
\end{equation}

The Doppler shift due to $v_{th}$ would be:

\begin{equation}
\nu'= (1 - \dfrac{\vec{v_{th}}\cdot\vec{n}}{c})\nu_{\alpha}
\end{equation}

Which can be expressed in terms of $\Delta \nu_D$ as:

\begin{equation}
\nu_{\alpha} - \nu' = \Delta\nu_D =  \dfrac{\vec{v_{th}}\cdot\vec{n}}{c} \nu_{\alpha}
\end{equation}

It is also common to express the line profile in terms of the velocity $V$
making use of Eq.\ref{eq:x}:

\begin{equation}
V = xv_{th} = \dfrac{\nu - \nu_{\alpha}}{\nu_{\alpha}}c
\end{equation}

\subsection{Average number of scatterings $N_{scatt}$}

The path of a \ly photon inside an {\bf{optically thick}}
HI medium is resonant this is explained in more detail 
in \S \ref{sec:resonant}. It basically means that the \ly photon
after been emitted by the source is absorbed by an H atom
and re-emitted in a time scale of $\sim 10^{-9}$ s.
For this reason, this process is commonly referred as
a scattering process. The total number of scatterings
is denoted as $N_{scatt}$ and it is proportional to
$\tau^2$.



\subsection{Resonant scattering}\label{sec:resonant}

Hydrogen is the most abundant element in the Universe furthermore
the \ly line is a very common line. However the transmission of these
\ly  photons through the Hydrogen clouds is not trivial. There 
are two main aspects that heavily influence the morphology of the line.

Imagine that there is one Hydrogen atom which emits 
a \ly photon in the middle of a cloud of HI. 
The \ly photon will be absorbed and re-emitted from 
the Hydrogen atom. This process will end up in a random walk in the space,
and the photon can escape the cloud at any point in of the surface, see 
Fig.\ref{fig:rw}.\\

\begin{figure}
\begin{center}
\includegraphics[scale=0.4]{Figures/randomwalk.png}
\end{center}\caption{Scattering scheme of a \ly photon in a HI medium
(The star represent a \ly source and the solid line represents the path that
the \ly photon follow before scaping the cloud).\label{fig:rw}}
\end{figure}


In the previous situation all the Hydrogen atoms were in rest, if the atoms
present proper motions there would be a Doppler shift Fig.\ref{fig:xshift}. 
If an atom have a velocity $v$ an emit a \ly photon in the direction 
opposite to the movement the \ly photon 
would be redshifted. But if it is emitted in the same direction of movement the
\ly photon would be blueshifted.    

\begin{figure}[H]
\begin{center}
\includegraphics[scale=0.4]{Figures/xshift.png}
\end{center}\caption{Scheme of the frequency shift. Left: Rest frame, Because 
of the velocity $u$ of the atom absorb a blue-shifted \ly photon, and emit 
a red-shifted \ly photon. Right HI atom rest frame when all photons have the \ly frequency. (Image credit: Interpereting Lyman $\alpha$ radiation from young, dusty galaxies. By: Peter Laursen, 2010.).\label{fig:xshift}}
\end{figure}


This two effects made ratiave trasnfer inside optically thick medium as a random 
walk in space and frequency. This is why Monte-Carlo methods can be
applyed to this diffusion process.
 
\subsection{Escape fraction}\label{sec:ef}

Dust grains are mainly considered as metals in the ISM, these
metals are formed in stars. At high redshifts where galaxies and 
stars are young the most probable escenario in that supernovae 
enrich the ISM with dust \citep{Kotak09}, in this way the observed 
dust \citep{Coppin09}  at high redshift is explained. 

The effect of dust in the \ly transfer inside the ISM is that 
dust grains can absorb (destroy) or scatter \ly photons. The 
probability of these events is given by the 'albedo' $A$ defined as:

\begin{equation}
A = \dfrac{\sigma_{scatt}}{\sigma_{dust}}
\end{equation}

Where $\sigma_{scatt}$ is the total cross section for scattering 
and $\sigma_{dust}$ for absorption. $\sigma_{dust}$ can be derived from 
dust properties see \citep{Laursen09}. 

There are two approaches to the modelling of dust in the \ly RT process.
As a first approximation the dust distribution is taken as an homogeneous distribution 
in the HI region and 
the \ly photons can be absorbed by the dust or scattered. A more
sophisticated situation is that the medium follows a clumpy distribution
of dust in which \ly photons are more likely to scatter with the 
clumps rather than absorbed \citep{Laursen13}. To quantify the effect of 
the dust the ratio
of \ly photons observed (\ly alpha photons who manage to scape from the medium) 
over the \ly photons emitted define a quantity call the {\bf{escape fraction}} $f_{esc}$. 

It is worth pointing out that the analytical
description of the \ly RT doesn't take into account the presence of
dust. That is also a motivation to include dust in the numerical simulations.
As is explained in more detail in \S \ref{sec:analytic}.  





\section{\emph{Attenuation of the \ly emission line}}

Despite the fact that the \ly line is the strongest emission line in 
the UV, there were 25 years since the prediction of the \ly line to 
the first observation by \citep{DjorgovskiThomson92}. 
This long absence of the \ly line is what makes this and excited and 
challenging field.

Figure\ref{fig:IGM} shows the outcoming spectra from a galaxy 
at $z \sim 3.5$, the line is double peaked as expected from 
the radiative transfer process in the galaxy. After the encounter
with the inter-galactic medium (IGM) de blue peak is diminished.
Dut to the expansion of the Universe the blue peak 
is shifted to the \ly frequency and the HI in the IGM would absorb
part of this radiation. As a result the observed spectra (right figure)
is asymmetric.   

\begin{figure}[H]
\begin{center}
\includegraphics[scale=0.6]{Figures/ISM.png}
\end{center}\caption{The IGM effect on the \ly line. (Image Credit:)\label{fig:IGM}}
\end{figure}


The gas kinematics in the galaxy also plays a mayor role in shaping
the morphology of the line, due to the resonant nature of the line. 
Figure\ref{fig:kulas} show the \ly line
profile for different galaxies at $z \sim 2 - 3$, due to the different
kinematics of those galaxies all the spectra reveals asymmetric profiles 
and some of them are multi-peaked.    


\begin{figure}[H]%\label{fig:kulas}
\begin{center}
\includegraphics[scale=0.4]{Figures/kulas.png}
\end{center}\caption{Asymmetry and multipeaked \ly line profiles. (Image credit: The kinematics of multiple-peaked \ly emission in star-forming galaxies at $z\sim 2$ - 3. Kulas, K. et al 2011)\label{fig:kulas}
 }
\end{figure}

The presence of dust in galaxies also play an important role, dust as explained
 in \S\ref{sec:ef} could either absorb or scatter \ly photons. As a 
consequence dust mainly diminish the intensity of the \ly line of LAEs. 

Alongside these processes when observing the abundance of LAEs across
the history of the Universe the population of LAEs increases 
with the redshift. But at redshifts $z>6$ the  
abundance of LAEs decrease \citep{Schenker12}. Apparently 
 reionization  is governing at that redshift and the IGM medium 
becomes opaque to \ly radiation.   

All these effects make the \ly line a sensitive line and challenging
to observe at $z>6$ see \citep{Sobral15}, but with valuable information
of the ISM/IGM distribution and kinematics. All these makes de \ly
line a very useful line to explore the extragalactic Universe as
discussed in \S\ref{sec:lyuses}.

  

\section{\emph{Analytical Models}}\label{sec:analytic}

Understanding the \ly line profile requires a theoretical knowledge
about the physics processes involving the radiative transfer. Simplified 
situations have been studied in order to obtain an analytical 
 profile.  In this section, the most relevant 
analytical models are explained in the chronological order in which 
they have been developed. 

The radiative transfer of \ly photons has been studied by several authors
see \citep{RybickiLightman79} for a complete treatment, the Intensity
of \ly photons can be studied via:
 
\begin{equation}\label{eq:RT}
n\cdot\nabla I(\nu, n)= - \alpha_{\nu} I(\nu, n) + j(\nu, n) + \int d\Omega' \int dn' I(\nu', n') R(\nu', \nu, n', n)
\end{equation}

Where $\nu$ is the frequency of the \ly photons. $n$ is the direction 
of the \ly photon. $I(\nu, n)$ is the intensity
of the radiation. $\Omega$ is the solid angle. $R(\nu', \nu, n', n)$ 
is the redistribution function, basically this function which measures 
the probability that a \ly photon with frequency $\nu'$ and direction $n'$
after scatters have a frequency $\nu$ and direction $n$.

The mean density $J_{\nu}$ is defined as:

\begin{equation}\label{eq:J}
J_{\nu} = \dfrac{1}{4\pi}\int I_{\nu}d\Omega
\end{equation}


Using Eq.\ref{eq:J} and following the above steps:

\begin{itemize}
\item In an optically thick medium the dependence on the direction {\bf{$n$}} can 
be neglected.
\item Using a Taylor expansion in ($I(\nu', n')$) in the second term of the right in Eq.\ref{eq:RT} 
\item Replacing the absorption coefficient in terms of the optical depth.
\item $j(\nu, n)=0$ and $\sigma_{dust}=0$
\end{itemize}

Eq.\ref{eq:RT} can be expressed as: 

\begin{equation}\label{eq:RTeq}
\dfrac{dJ(\nu)}{d\tau} = \dfrac{(\Delta \nu_D)^2}{2}\dfrac{\partial}{\partial \nu}\phi(\nu)\dfrac{\partial J(\nu)}{\partial \nu}
\end{equation}

Where $\phi(\nu)$ is a Voigt profile (Combolution of a Gaussian profile  
and Lorentzian profile), this Voigt profile respond to the resonant nature 
of the line. Eq.\ref{eq:RTeq} is a diffusion equation in space and frequency
for the \ly photons in HI clouds. Different authors have solved 
Eq.\ref{eq:RT} in simplified situations that we are going to discuss above. 

\subsection{Infinite Slab with \ly source at the center:}

The first analytical solution to the RT Eq.\ref{eq:RTeq}
was an effort in which different authors made a contribution \citep{Unno55, Osterbrock62, Adams72, Harrington73} and ended with the analytical expression
derived by \citep{Neufeld90} based on the previous works. 

\begin{equation}
J(\tau, x) = \dfrac{\sqrt{6}}{24}\dfrac{x^2}{\sqrt{\pi}a\tau cosh[\sqrt{\pi^3/54}(x^3-x_{in}^3)/a\tau]}
\end{equation}

Where $a$ is the Voigt parameter defined as $a=A/4\pi\Delta \nu_D$. \citep{Harrington73} also show that the maximum intensity of the line is at:

\begin{equation}
x_m = \pm1.066(a\tau)^{1/3}
\end{equation}

And the average number of scatterings is:

\begin{equation}
N_{scatt} = 1.612\tau
\end{equation}

In Fig.\ref{fig:slab} the analytical profile of the slab solution is shown, 
the solid line is the simulated profile reproduced with \citep{CLARA} the code
used in this work.

\begin{figure}[H]
\begin{center}
\includegraphics[scale=0.4]{Figures/slab.png}
\end{center}\caption{(left) Analytic profile for a dustless infinite slab with central
\ly sources. (Middle) Maximum peaks position. (Right) Average number of 
scatterings $N_{scatt}$ in function of the optical depth $\tau$.(Image credit:  CLARA's view on the escape fraction of \ly photons in high redshift galaxies. J.E Forero-Romero, et al 2011)\label{fig:slab}}
\end{figure}

\subsection{Spherical solution}

For a spherical gas dustless distribution with central \ly sources \citep{Dijkstra06} has shown that the emergent spectrum is described by the following expression:

\begin{equation}
J(\tau, x) = \dfrac{\sqrt{\pi}}{4\sqrt{6}}\dfrac{x^2}{a\tau (1+cosh[\sqrt{2\pi^3/27}x^3/a\tau])}
\end{equation}

The analytical spectrum of the spherical solution 
is shown in Fig.\ref{fig:sphere} for different optical depths. 

\begin{figure}[H]
\begin{center}
\includegraphics[scale=0.4]{Figures/Sphere.png}
\end{center}\caption{Analytic profile for a dustless sphere with central \ly sources.  (Image credit: CLARA's view on the escape fraction of \ly photons in high redshift galaxies. J.E Forero-Romero, et al 2011)\label{fig:sphere}}
\end{figure}

\subsection{Spherical rotating sphere}

In this work Mark Dijkstra has shown also that an approximate analytical profile
can be derived for a {\bf{rotating spherical distribution}} Eq.\ref{eq:spherero}
with central sources and dustless. This would be the first time that an analytic
al profile is derived for a non-static medium.  

\begin{equation}\label{eq:spherero}
J(x,b,\phi,i)=\frac{\sqrt{\pi}}{\sqrt{24}a\tau_0}\Bigg{(}\frac{(x-x_{\rm
b})^2}{1+{\rm cosh}\Big{[}\sqrt{\frac{2\pi^3}{27}}\frac{|(x
-x_{\rm b})^3|}{a\tau_0}\Big{]}}\Bigg{)}
\end{equation}

\begin{equation}
J(x,i)= \approx 2\pi \int_0^Rdb \hs b
\int_0^{2\pi}d\phi \hs J(x,b,\phi,i)\\ \nonumber
\end{equation}

\section{\emph{Simulated models \& techniques}}

In the Universe most of the galaxies have complicated
geometries (Spirals, irregular) and kinematics that 
can not be resolved analitically. There are two approaches
to understant these complicated properties of galaxies, based
on Monte-Carlo simulations. The first approach is implementing 
the properties of the galaxies, such as: The geometry
of the gas, the kinematics of the gas and the dust. In this 
approach every property is isolated simulated in order
to compute the \ly profiles, escape fractions, average number
of scatterings, position of the maximum peaks among others. 

Realistic galaxy properties in the sense of gas geometry
and kinematics using hydrodynamic simulations. In this 
approach the galaxy can be simulated isolated see \citep{Verhamme12}
or galaxies in the cosmic web can also be studied see\citep{Yajima12}.
The temperature, density and gas velocity fields are obtained from 
the hydrodynamic simulations and then the Monte-Carlo code is 
implemented in order to compute the profile properties.

\subsection{Monte-Carlo approach:}

Monte-Carlo (MC) simulations are broadly used in a variety of 
areas such as science, economy, traffic simulations among 
many more. MC methods are mostly based in the
generation of random numbers that are the core of 
random walks. This is why MC is used to study the
\ly profile. Here we briefly describe how this method 
work. The main thing is that every every \ly photon is simulated
separately in the following scheme, for a detailed description 
please see chapters $6-8$ in \citep{LaursenPhD} .

\begin{itemize}
\item  The temperature (T) is very common to take $T=10^4K$, gas distribution ($\tau$) and kinematics ($V$) is implemented. 

\item Initialize your \ly photon initial position and frequency $x_{in}$

\item Generate a random displacement ($\tau_0$) of the photon in a random 
direction {\bf{$\vec{n}$}}.

\item  Derive the HI atom velocity components from the initial field, 
and generate random components for the thermal movements.

\item  Set the new direction of the \ly photon after scattering.

\item If the \ly photon encounter a dust particle the albedo probability 
would define if the atom is absorbed or scattered. In also common to take $A=1/2$

\item Repeat from step 2 untill the photon reaches the HI surface at $\tau$.  

\end{itemize}

With this method the final frequency $x_{out}$, the average number of scatterings $N_{scatt}$ can be computed for every step in the random walk.


\subsection{Numerical Models of \ly profiles:}

Using the MC method explained above different RT  codes 
\citep{DijkstraKramer, Laursen09, Verhamme06, CLARA}
have been developed in order to understand the effect of the gas kinematics in
the \lya line, expanding/contracting shell/spherical geometries
has been broadly studied \citep{Ahn03,Verhamme06,Dijkstra06}.
Realistic ISM/IGM medium has also been studied, the effect of a clumpy
medium is discussed in \citep{Hansen06}. Anisotropic \ly emission 
has been studied by \citep{Zheng2013}. Realistic expanding mediums 
in cavities has been recently studied by  \citep{Behrens2014} 
Hydrodynamic simulations have studied the outcoming spectra of
LAEs in large scale simulations \cite{Forero12}. 
Recently Monte Carlo codes have been used in hydrodynamic 
simulations to study in detail individual galaxies and galaxies 
in the cosmic web.
\citep{Laursen09,Barnes11,Verhamme12,Yajima12}

Figure \ref{fig:out} shows the effect of outflow/inflow kinematics, 
In the outflow regime photons are blueshifted and the blue part of the line
is stronger than the red part. In the inflow regime the opposite effect
is carried out, the \ly photons are redshifted. 

\begin{figure}[H]%\label{fig:out}
\begin{center}
\includegraphics[scale=0.4]{Figures/out.png}
\end{center}\caption{\ly profile of outflows (Expanding medium) dot line, 
and inflows (Contracting) dot-line. (Image credit: CLARA's view on the escape fraction of \ly photons in high redshift galaxies. J.E Forero-Romero, et al 2011)\label{fig:out}
 }
\end{figure}

\section{\emph{This Thesis}}

In this thesis we study the effect of rotation, an intrinsic
 characteristic of galaxies, on the morphology
of the \lya outcoming profile, to this aim we implement a solid body
rotation model in the radiative transfer code \verb+CLARA+ \citep{CLARA}.
All the codes used for the analysis of this work is reproducible
and public available in \href{https://github.com/jngaravitoc/RotationLyAlpha}{github}.

\subsection{The importance of modelling the effect of gas bulk rotation}

Untill now the effect of rotation have never been studied, and this is 
an intrinsic properties of all galaxies. In this work we study for the 
first time the effect of galaxy rotation the morphology of the \ly line.

Modelling the effect of rotation in the morphology of the \ly line, push
further our understanding of the kinematics effects on the \ly line. 
With these models a realistic analysis can be made in observed \ly spectra. 
Now a distinction between the different stages of the gas kinematics
 outflow/inflow, anisotropic \ly emission, shell cavities and now
 rotation is possible.

Deriving and analytic expression for rotation is also important for
the community, computing an analytic solution is more efficient than
making all the radiative transfer simulation. Radiative transfer codes
can be tested against the analytic solution and observed spectra could
be easily fitted  with the model. 

\subsection{Summary of the thesis}

A lot of progress has been done in modelling the \ly line using radiative
transfer codes. Properties of the gas kinematics such as outflows/inflows. 
Geometries such as slabs, spheres, cavities and the propagation of \ly photons
in a clumpy media. In this thesis we study the effect 
of rotation in the morphology of the \ly profile.  

We model a galaxy as a sphere, with an homogeneous mixture of gas and dust. 
We took the rotation velocity, the optical depth, the viewing angle as free
parameters of the model. We also have two different sources of \ly photons:
A central distribution in the galaxy and an homogeneous distribution. 

We quantify the effect of rotation with the following characteristics: 
Escape fraction of \ly photons, the width of the \ly line, the average number
of scatterings of the \ly photons and with the position of the \ly line maximum.

Our main finding is that rotation do have an important effect on the morphology 
of the \ly line. Specially in the width of the line and in the position of the 
maxima. While the average number of scatterings  and the scape
fraction remain constant.

The line broadens proportional to the rotation velocity, and also the flux in 
the middle of the line increases with rotation.

The axis of rotation breaks the symmetry of the system, Althogh 
observers in different viewing angle with respect to the rotation axis
observe the same amount of flux of the \ly profile. But the morphology
of the profile is affected with the observer position, at some angles
the profile is double-peaked and others have single peaked profiles. 
 
 
With these results an approximated analytical solution was derived, 
taking into the account that the radiative transfer inside de gas
cloud is exactly as in the static case. In the sphere surface
a Doppler shift due to the difference velocity of an external 
observer and the surface have to be taken into account.  


\subsection{Future work}


There are two main projects which are based on the results obtained in 
this work. With the aim of testing our model with observations and
improving the kinematic models of the gas by studying two joint effects
such as outflows and rotation.  

\begin{itemize}


\item There are LAEs which are governed by rotation, also 
have approximated spherical geometries and the main \ly sources are in the center. 
With these properties such galaxies have the same properties 
that we studied in this work.  
Fitting \ly line profile in order  to measure the rotational 
velocities of such galaxies would be the direct application 
and test of our model. 

To this aim we are fitting our model using a Marov Chain Monte-Carlo 
method to the observed compact dwarf galaxy \verb+TOL1214-277+\citep{Thuan97, Verhamme15}
. The spectra of this galaxy is shown in Fig.\ref{fig:tol}, 
this galaxy have properties that lead us to derive the rotational 
velocity; it is a compact dwarf galaxy whose geometry can be approximated by a sphere, 
is young and there is no evidence of outflows see \citep{Verhamme15} for a detailed
discussion, in that paper this triple-peaked spectum could not be explained
with the current radiative transfer models. While in our rotation 
model is a common feature to see a triple-peaked spectrum. 

\begin{figure} 
\begin{center}
\includegraphics[scale=0.6]{Figures/tol.png}
\end{center}\caption{Tol1214-277 spectra.\label{fig:tol}} 
\end{figure}

\item Despite the fact that outflows have been broadly studied rotation should 
also be present on these galaxies. The joint effect of the two above properties 
should have a direct effect on the morphology of the \lya line. We are 
 combining the effect of rotation followed by an outflow. For the rotation
part we are using our analytic solution and for the outflows
we are using the shell described in \citep{Verhamme12}. 
\end{itemize}
