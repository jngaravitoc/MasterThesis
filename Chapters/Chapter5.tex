% Chapter Template

\chapter{Conclusions} % Main chapter title

\label{sec:conclusions} % Change X to a consecutive number; for referencing this chapter elsewhere, use \ref{ChapterX}

\lhead{\emph{Conclusions}} % Change X to a consecutive number; this is for the header on each page - perhaps a shortened title

%----------------------------------------------------------------------------------------
%	SECTION 1
%----------------------------------------------------------------------------------------

In this work we quantified for the first time in the literature the effects
of gas bulk rotation in the morphology of the \ly emission line in
star forming galaxies.
Our results are based on the study of an homogeneous sphere
of gas with solid body rotation.
We explore a range of models by varying the rotational speed, hydrogen
optical depth, dust optical depth and initial distribution of \ly
photons with respect to the gas density.
As a cross-validation, we obtained our results from two independently
developed Monte-Carlo radiative transfer codes.
Two conclusions stand out from our study.
First, rotation clearly impacts the \ly line morphology; the width and
the relative intensity of the center of the line and its peaks are
affected.
Second, rotation introduces an anisotropy for different viewing
angles.
For viewing angles close to the poles the line is double peaked and it
makes a transition to a single peaked line for high rotational
velocities and viewing angles along the equator.
This trend is clearer for spheres with homogeneously distributed
radiation sources than it is for central sources.
Remarkably, we find three quantities that are invariant with respect
to the viewing angle and the rotational velocity: the integrated flux,
the escape fraction and the average number of scatterings.
These results helped us to construct the outgoing spectra of a
rotating sphere as a superposition of spectra coming from a static
configuration. This description is useful to describe the main
quantitative features of the Monte Carlo simulations.
Quantitatively, the main results of our study are summarized as
follows.
\begin{itemize}
\item In all of our models, rotation induces changes in the line morphology
for different values of the angle between the rotation
axis and the LoS, $\theta$. The changes are such that for
a viewing angle perpendicular to the
rotation axis, and high rotational velocities the line becomes single peaked.
\item The line width increases with rotational
velocity. For a viewing angle perpendicular to the rotation axis
This change approximately follows the functional form ${\rm FWHM}^2
= {\rm FWHM}_{ 0}^2 + (V_{\rm max}/\lambda)^2$, where FWHM$_{0}$
indicates the line
width for the static case and $\lambda$ is a constant. We have
determined this constant to be $\lambda_{\rm c}=0.83 \pm 0.06$ and
$\lambda_{\rm h}=0.82\pm 0.05$ for the central and homogeneous source
distributions, respectively.
\item At fixed rotational velocity the line width decreases as $|\mu|$
increases, i.e. the smallest value of the line width is observed for
a line of sight parallel to the ration axis.
\item The single peaked line emerges at viewing angles $\mu\sim 1$ for
when the rotational velocity is close to than half the FWHM$_0$.
\end{itemize}
Comparing our results with recent observed LAEs we find that
morphological features such as high central line flux, single peak
profiles could be explained by gas bulk rotation present in these
LAEs.
The definitive and clear impact of rotation on the \ly morphology
suggests that this is an effect that should be taken into account at
the moment of interpreting high resolution spectroscopic data. In
particular it is relevant to consider the joint effect of rotation the
and ubiquitous outflows (M.C. Remolina-Gutierrez et al., in prep.)
because rotation can lead to enhanced escape of \ly at line center, which
has also been associated with escape of ionizing (LyC) photons
\citep{Behrens2014,2014arXiv1404.2958V}
